\subsection{Del a} % (fold)
\label{sub:del_a}
Da ladningen ligger i overfladen af metalpladen vil den Gaussflade med størst toppladeareal omslutte den største ladning. Rækkefølgen vil derfor være $S_3, S_2, S_1$.
% subsection del_a (end)

\subsection{Del b} % (fold)
\label{sub:del_b}
Vi har formlen for størrelsen det elektriske felt ved endefladen på et legeme halvt nedsænket i en leder:
\begin{equation}
	E = \frac{\sigma}{\epsilon _0}
\end{equation}
Da overfladearealet ikke indgår i formlen har det ikke nogen indflydelse. Størrelsen af det elektriske felt ved topfladerne må derfor være det samme for alle tre Gaussflader.
% subsection del_b (end)

\subsection{Del c} % (fold)
\label{sub:del_c}
Det elektriske felt inde i en leder er lig nul og derfor må det elektriske felt igennem bundfladerne også være lig nul.
% subsection del_c (end)