%!TEX root = /Users/tamen/Documents/SDU/EFYS/Opgaver/EFYS--Gauss-Lov/main.tex
\subsection{Del a} % (fold)
\label{sub:del_a}
Fluxen igennem bundfladen vil være det samme som fluxen igennem topfladen, dog med omvendt fortegn. Fluxen gennem sidevæggen vil være lig nul, da væggen er parallel med det elektriske felt. Den elektriske flux vil derfor være lig nul. Endefladerne ophæver hinanden, og sidevæggen  bidrager ikke. Den elektriske flux vil derfor være det samme for dem alle sammen og lig nul.
% subsection del_a (end)

\subsection{Del b} % (fold)
\label{sub:del_b}
Den elektriske flux gennem topfladerne er også ens. Jo mere skrå en flade det elektriske felt strømmer igennem, jo mindre er fluxen for det enkelte del-areal. Til gengæld er der flere del-arealelementer der sakl lægegs sammen for at finde den samme flux, hvilket modvirker den mindre flux for de enkelte skrå del-arealer.

En analogi kan være en ventilationsskakt med strømmende luft. Uanset den flade du måler luftgennemstrømningen igennem, vil der stadig strømme den samme mængde luft gennem skakten.
% subsection del_b (end)