%!TEX root = /Users/tamen/Documents/SDU/EFYS/Opgaver/EFYS--Gauss-Lov/main.tex
\subsection{Del a} % (fold)
\label{sub:del_a}
Hvis vi skal have ophævet feltet i punkt 1 (som jeg vælger at tolke som liggende lige langt fra overfladen af cylinder $a$ og den inderste overflade af cylinder $b$) skal indersiden af cylinder $b$ have den samme ladning som overfladen af cylinder $a$. Dette vil gøre at feltlinjerne afbøjes, så feltet er lig nul i punkt 1. 
% subsection del_a (end)

\subsection{Del b} % (fold)
\label{sub:del_b}
Hvis indersiden af cylinder $b$ har ladningen $+3q_0$ så vil ydersiden have ladningen $-3q_0$ hvis vi så sætter indersiden af cylinder til at have ladningen $-3q_0$ vil feltet i punkt 2 være lig nul.
% subsection del_b (end)

\subsection{Del c} % (fold)
\label{sub:del_c}
Uanset hvilken ladning vi giver cylindrene vil der altid være et felt fra ladningerne inde i cylindrene. Vi kan derfor ikke komme af med det elektriske felt i punkt 3.
% subsection del_c (end)