Kraften på en testmasse i et tyngdefelt er:
\begin{equation}
	\vec{F} = G\cdot \frac{m \cdot m_0}{r^2} \cdot \vec{r}
\end{equation}
Det ligner til forveksling kraften for en punktladning i et elektriskfelt:
\begin{equation}
	\vec{F} = k \cdot \frac{q \cdot q_0}{r^2} \cdot \vec{r}
\end{equation}
For at finde tyngdefeltet, $\vec{T}$, skal vi dividere kraften med testmassen:
\begin{equation}
	\vec{T} = \frac{\vec{F}}{m_0} = G \cdot \frac{m}{r^2} \cdot \vec{r}
\end{equation}
Hvilket er udtrukket for tygndefeltet opstillet på samme måde som Gauss' lov:
\begin{equation}
	\vec{E} = k \cdot \frac{q}{r^2} \cdot \vec{r}
\end{equation}