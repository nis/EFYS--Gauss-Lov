%!TEX root = /Users/tamen/Documents/SDU/EFYS/Opgaver/EFYS--Gauss-Lov/main.tex
Kraften på en testmasse i et tyngdefelt er:
\begin{equation}
	\vec{F} = G\cdot \frac{m \cdot m_0}{r^2} \cdot \vec{r}
\end{equation}
Det ligner til forveksling kraften for en punktladning i et elektriskfelt:
\begin{equation}
	\vec{F} = k \cdot \frac{q \cdot q_0}{r^2} \cdot \vec{r}
\end{equation}
For at finde tyngdefeltet, $\vec{T}$, skal vi dividere kraften med testmassen:
\begin{equation}
	\label{eqn:tyngdefelt}
	\vec{T} = \frac{\vec{F}}{m_0} = G \cdot \frac{m}{r^2} \cdot \vec{r}
\end{equation}
Hvilket er udtrukket for tygndefeltet opstillet på samme måde som Gauss' lov:
\begin{equation}
	\vec{E} = k \cdot \frac{q}{r^2} \cdot \vec{r}
\end{equation}

Hvis tyngdefeltet skal beregnes hvor testmassen ligger uden for jorden er det bare ligning \ref{eqn:tyngdefelt} der skal bruges.
Skal tyngdefeltet beregnes inden for jorden, skal der tages højde for at det ikke er hele jordens massen der påvirker testmassen. Da vi snakker om Gauss flader vi "intelligent" har lagt så tyngdekraftvektorerne ligger vinkelret på overfladen af Gaussfladen vil tyngdefeltet fra den del af jordens masse der ligger uden for Gaussfladen være lig nul, da tyngdekraften på en del af Gaussfladen ophæves af tyngdekraften på den dimentralt modsatte del af Gaussfladen.
Ergo kan vi opsætte formlen således:
\begin{equation}
	\label{eqn:tyngdefelt2}
	T = \frac{\vec{F}}{m_0} = G \cdot \frac{m'}{r^2}
\end{equation}
Hvor $m'$ er den del af jordens masse der ligger inden for Gaussfladen. Hvis vi antager at tyngden i jorden er ligeligt fordelt kan vi sige at forholdet mellem hele tyngden af jorden og dens volumen, må være lig forholdet mellem tyngden af den del af jorden inden for Gaussfladen og volumen af Gaussfladen:
\begin{eqnarray}
	\frac{m'}{\frac{4}{3}\pi r^3} &=& \frac{m}{\frac{4}{3}\pi R^3} \Rightarrow \nonumber \\
	m' &=& q\frac{r^3}{R^3}
\end{eqnarray}
Hvor $r$ er Gaussfladens raidus og $R$ er jordens radius. Putter man det ind i ligning \ref{eqn:tyngdefelt2} får man:
\begin{equation}
	E = (\frac{q}{4\pi \epsilon _0 R^3})r
\end{equation}