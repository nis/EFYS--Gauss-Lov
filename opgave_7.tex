%!TEX root = /Users/tamen/Documents/SDU/EFYS/Opgaver/EFYS--Gauss-Lov/main.tex
Den inderste kugle har ladningen $q$ den inderste skal vil så få en ladning på indersiden på $-q$ og ladningen på ydersiden vil så være $4q$ da nettoladningen skal være $3q$. Den ydersteskal vil så få en ladning på indersiden på $-4q$ og på ydersiden $9q$ da nettoladningen skal være $5q$. Vi får så Elektriske felter for Gaussfladerne, $G_1, G_2, G_3$ på:
\begin{eqnarray}
	E_{G1} &=& k \cdot \frac{q}{r^2} \nonumber \\
	E_{G2} &=& k \cdot \frac{4q}{(2r)^2} = k \cdot \frac{4q}{4r^2} = k \cdot \frac{q}{r^2} \nonumber \\
	E_{G3} &=& k \cdot \frac{9q}{(3r)^2} = k \cdot \frac{9q}{9r^2} = k \cdot \frac{q}{r^2} \nonumber \Rightarrow \\
	E_{G1} &=& E_{G2} = E_{G3}
\end{eqnarray}
Som det kan ses er størrelsen af det elektriske felt på de tre Gaussflader ens.
