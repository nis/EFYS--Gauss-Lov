%!TEX root = /Users/tamen/Documents/SDU/EFYS/Opgaver/EFYS--Gauss-Lov/main.tex
\subsection{Del a} % (fold)
\label{sub:del_a}
E-feltet når $r = 0$ er lig nul da der ikke er nogen ladning inde i vores Gaussflade.
% subsection del_a (end)

\subsection{Del b} % (fold)
\label{sub:del_b}
E-feltet når $r = \frac{a}{2}$ er lig nul da der ikke er nogen ladning inde i vores Gaussflade.
% subsection del_b (end)

\subsection{Del c} % (fold)
\label{sub:del_c}
E-feltet når $r = a$ er lig nul da der ikke er nogen ladning inde i vores Gaussflade.
% subsection del_c (end)

\subsection{Del d} % (fold)
\label{sub:del_d}
E-feltet kan udregnes ved følgende formel:
\begin{equation}
	E = \frac{1}{4\pi \epsilon _0} \cdot \frac{q}{r^2}
\end{equation}
Hvor $q$ er den ladning Gausfladen omslutter og $r$ er radius af Gaussfladen, der i dette tilfælde er en kugle.

Sætter vi $r$ til $1.5a$ vil Gaussfladen omslutte noget ladning. Rumfanget af den del af kuglen der har en ladning må være:
\begin{equation}
	V = V_d - V_a
\end{equation}
Hvor $V_a$ er rumfanget af den kugle uden ladning og $V_d$ er rumfanget af kuglen med radius $1.5a$. Resultatet, $V$, vil så kunne bruges til at regne ud hvor maget ladning der er indenfor Gaussfladen da vi har ladningsdensiteten for objektet: $\rho = 1.84 \cdot 10^{-9}\frac{C}{m^3}$

Ladningen er:
\begin{eqnarray}
	q &=& \rho \cdot V \\
	q &=& 1.84 \cdot 0.00994838 \\
	q &=& 1.8305 \cdot 10^{-11} C. 
\end{eqnarray}

Når vi har fundet ladningen kan vi regne E-feltet ud:
\begin{eqnarray}
	E &=& \frac{1}{4\pi \epsilon _0} \cdot \frac{q}{r^2} \\
	E &=& 8.99\cdot 10^9 \cdot \frac{1.8305}{0.15^2} \\
	E &=& 7.31387 
\end{eqnarray}
% subsection del_d (end)

\subsection{Del e} % (fold)
\label{sub:del_e}
Fremgangsmåden er præcist den samme som i del d, man sætter bare $r = b$.
\begin{equation}
	E = 12.1256
\end{equation}
% subsection del_e (end)

\subsection{Del f} % (fold)
\label{sub:del_f}
Her skal man passe lidt på da den ladning man skal regne på ikke fylder helt ud til $r = 0.6m$, her vild en ladning Gaussfladen indeholde være lig $5.39516C$ (Ladningen vil være ladningen af kuglen med $r = 0.2m$ minus det indre hulrum). Så vil E-feltet blive:
\begin{equation}
	E = 1.34729
\end{equation} 
% subsection del_f (end)