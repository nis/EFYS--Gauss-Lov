%!TEX root = /Users/tamen/Documents/SDU/EFYS/Opgaver/EFYS--Gauss-Lov/main.tex
Vi har formlen for netto flux gennem en lukket flade:
\begin{equation}
	\Phi = \frac{q_{enc}}{\epsilon _0}
\end{equation}
Da en terning er lavet af seks lige store dele, må fluxen gennem en side være en sjettedel af den samlede flux:
\begin{equation}
	\Phi_{\frac{1}{6}} = \frac{q}{\epsilon_0} \cdot \frac{1}{6}
\end{equation}