%!TEX root = /Users/tamen/Documents/SDU/EFYS/Opgaver/EFYS--Gauss-Lov/main.tex
\subsection{Del a} % (fold)
\label{sub:del_a}
Uden at have noget udtryk for det vil jeg mene at den eneste pertikel der ikke bidrager med et elektrisk felt i punktet $P$ er $q_3$, da de tre andre partikler "spærrer" for dens feltlinjer ved at have samme ladning og derved skubber dem væk. Dog kan der argumenteres for at $q_3$ vil have en indflydelse på de tre andre partiklers feltlinjer, og at $q_3$ derved også har en indflydelse på det elektriske felt i punktet $P$
% subsection del_a (end)

\subsection{Del b} % (fold)
\label{sub:del_b}
Netto fluxen gennem en Gaussflade er defineret ved summen af de indkapslede ladninger delt med $\epsilon _0$, ergo er fluxen fra $q_1$ og $q_2$ størst:
\begin{equation}
	\Phi = \frac{q_1 + q_2}{\epsilon _0}
\end{equation}
% subsection del_b (end)